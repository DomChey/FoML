\documentclass[11pt]{report}

\usepackage[utf8]{inputenc}
\usepackage{algorithm}
\usepackage{algpseudocode}
\usepackage{hyperref}
\usepackage{graphicx}
\usepackage{subcaption}
\usepackage{booktabs}
\usepackage[square,sort,comma,numbers]{natbib}
\bibliographystyle{abbrv} 

% Title Page
\title{\textbf{A Neural Network to solve square jigsaw puzzles}}
\author{Advanced Machine Learning \\ \\
  Project Report by \\
  Dominique Cheray and Manuel Krämer}

\begin{document}
\maketitle

\tableofcontents

\chapter{Introduction}
\subsubsection*{by Dominique Cheray}
Solving jigsaw puzzles is a pastime that children all over the world know well.
Given a set of often oddly shaped interlocking pieces with small parts of a
picture on top of them the goal is to assemble the pieces in the correct way to
reconstruct the full image. But solving puzzles is not only a fun pastime it
also finds its applications in areas such as archaeology \cite{brown2008system,
  liu2011automated, koller2006computer}, biology
\cite{marande2007mitochondrial}, speech descrambling \cite{zhao2007puzzle},
image editing \cite{cho2008patch} or the reconstruction of fragmented documents
\cite{zhu2008globally}.

The first automatic jigsaw puzzle solvers focus only on the shape of the pieces
to solve the puzzles \cite{freeman1964apictorial, wolfson1988solving,
webster1990computer, kong2001solving}. Later on approaches are presented, which
take into account the color information in addition to the shape of the pieces
\cite{kosiba1994automatic, makridis2006new, sagiroglu2006texture}. Recent work
began to focus on only using the color information of the pieces
\cite{nielsen2008solving} which then eventually shifted to only consider jigsaw
puzzles with square pieces where color information is the only information that
can be used to find matching pieces \cite{Cho2010, yang2011particle,
  Pomeranz2011, gallagher2012jigsaw, son2014solving, sholomon2013genetic,
  Paikin2015, sholomon2016dnn}.

The first to introduce square pieces are Cho et al. \cite{Cho2010}. They present
a probabilistic solver that can handle puzzles of up to 432 pieces. To solve the
puzzle the solver needs some apriori knowledge of the puzzle, namely its size
and the position of few so-called ``anchor-pieces'' which are placed at their
correct location prior to the placement of the remaining pieces. A year later
the results of Cho et al. are improved by Yang et al. \cite{yang2011particle}
which use a particle filter based solver.

Pomeranz et al.\cite{Pomeranz2011} are the first to introduce a fully automatic
jigsaw puzzle solver. Their solver is based on a greedy placer and can handle
puzzles of up to 3,300 pieces. Later on the work of Pomeranz et al. is improved
and extended by both Gallagher et al. \cite{gallagher2012jigsaw} and Son et al.
\cite{son2014solving}. Gallagher use a greedy tree algorithm and generalize the
work of Pomeranz et al. to also handle pieces of unknown orientation and puzzles
with unknown dimensions. Son et al. add ``loop constraints'' to the work of
Gallagher et al. which allows them to handle pieces of unknown orientation.

Instead of a greedy solver Sholomon et al. \cite{sholomon2013genetic} present a
genetic algorithm that is able to solve large square jigsaw puzzles. Later on
they improve their work by introducing a deep neural-network based estimation
metric \cite{sholomon2016dnn}. Given the edges of two pieces the neural network
predicts whether the two pieces belong next to each other in the correctly
assembled puzzle or not. The authors state that their metric shows
extremely high precision without the need of manual feature extraction. When
integrated into an existing solver it will significantly improve the results
\cite{sholomon2016dnn}. 

The solver presented by Paikin \& Tal \cite{Paikin2015} is inspired by the work
of Pomeranz et al. \cite{Pomeranz2011}. Their algorithm is also a greedy solver
but is able to solve more challenging puzzles. It can handle jigsaw puzzles with
missing pieces, pieces of unknown orientation, puzzles of unknown size and
multiple puzzles whose pieces are mixed together. The placement of the puzzle
pieces is based on a compatibility function but they provide a faster and more
accurate function than previous works. Additionally, since early mistakes can
have a fatal impact when using a greedy placer, they take special care when
choosing the first piece to place. Earlier works randomly select the first
piece. Another change made by Paikin \& Tal is to place the pieces in relation
to the pieces already placed and not on absolute positions. So the next piece to
place is not the piece that best fits a particular spot, but the most likely to
be correct.

For this project we will integrate the neural-network based estimation metric
proposed by Sholomon et al. \cite{sholomon2016dnn} in our reimplementation of the
solver by Paikin \& Tal \cite{Paikin2015} from last semester's project. We evaluate the resulting
solver with the same image datasets as the authors and compare our result to
theirs as well as to the results from our implementation of Paikin \& Tal's
solver from last semester.

\bibliography{literature}

\end{document}    
