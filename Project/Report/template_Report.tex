\documentclass[]{report}

\usepackage[utf8]{inputenc}

\bibliographystyle{alpha}

% Title Page
\title{Solving multiple square jigsaw puzzles with missing pieces}
\author{Dominique Cheray and Manuel Krämer}

\begin{document}
\maketitle

\begin{abstract}
\end{abstract}

\chapter{Introduction}

\chapter{Theoretical Background}
\subsubsection*{by Dominique Cheray}
\section{Jigsaw Puzzles}
The first jigsaw puzzles were produced around 1760 by John Spilsbury, a
London engraver and mapmaker, and made out of wood \cite{sholomon2013genetic}.
Hence the name ``jigsaw'' which refers to the jigsaws that were used to cut out
the pieces of the puzzles. Modern jigsaws puzzles, where an image was printed on
a cardboard sheet that was cut into a set of interlocking pieces were, introduced
in the 1930s \cite{williams2004jigsaw}. Even though puzzles are successfully
solved by children worldwide, automatic puzzle solvers are a technically
challenging problem. Demaine et al. \cite{demaine2007jigsaw} could show that the
puzzle problem is NP-complete if the pairwise affinity among pieces is
unreliable. \\
The first automatic jigsaw puzzle solver was proposed in 1964 by Freeman et al
\cite{freeman1964apictorial}. It was designed to solve jigsaw puzzles with
pieces which are all uniformly gray and the only available information is the
shape of the pieces. 

\chapter{Materials and Methods}

\chapter{Results}

\chapter{Discussion}

\nocite{*}

\newpage
\addcontentsline{toc}{section}{References}
\bibliography{literature}

\end{document}          
